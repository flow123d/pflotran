
\section*{Appendix D: Python Tools}
\addcontentsline{toc}{section}{\protect\numberline{}{\bf Appendix D: Python Tools}} 

\setcounter{section}{4}
\setcounter{subsection}{0}
\setcounter{equation}{0}
\setcounter{table}{0}
\setcounter{figure}{0}

\subsection{Extracting Aqueous Secondary Species, Minerals and Gases from a Thermodynamic Database}
%\addcontentsline{toc}{subsection}{\protect\numberline{}{\bf Extracting Aqueous Secondary Species, Minerals and Gases}} 

A python script is available to help the user extract secondary species, gases and minerals from the thermodynamic database for a given set of primary species. Surface complexation reactions are not included. The python script can be found in {\tt ./tools/contrib/sec\_species/rxn.py} in the PFLOTRAN mercurial repository. 
The current implementation is based on the {\tt hanford.dat} database.
Input files are {\tt aq\_sec.dat}, {\tt gases.dat} and {\tt minerals.dat}. In addition, for each of these files there is a corresponding file containing a list of species to be skipped: {\tt aq\_skip.dat}, {\tt gas\_skip.dat} and {\tt min.dat}.
Before running the script it is advisable to copy the entire directory {\tt sec\_species} to the local hard drive to avoid conflicts when updating the PFLOTRAN repository. To run the script simply type in a terminal window: 

{\tt python rxn.py}

\noindent
The user has to edit the {\tt rxn.py} file to set the list of primary species. For example,

{\tt pri=['Fe++','Fe+++','H+','H2O']}

\noindent
Note that the species H2O must be include in the list of primary species.
Output appears on the screen and also in the file {\tt chem.out}, a listing of which appears below. The number of primary and secondary species, gases and minerals is printed out at the end of the {\tt chem.out} file.

\noindent
{\tt chem.out} 

\footnotesize
\begin{verbatim}
PRIMARY_SPECIES
Fe++
Fe+++
H+
H2O
/
SECONDARY_SPECIES
O2(aq)
H2(aq)
Fe(OH)2(aq)
Fe(OH)2+
Fe(OH)3(aq)
Fe(OH)3-
Fe(OH)4-
Fe(OH)4--
Fe2(OH)2++++
Fe3(OH)4(5+)
FeOH+
FeOH++
HO2-
OH-
/
GASES
H2(g)
H2O(g)
O2(g)
/
MINERALS
Fe
Fe(OH)2
Fe(OH)3
FeO
Ferrihydrite
Goethite
Hematite
Magnetite
Wustite
/
================================================
npri =  4  nsec =  14  ngas =  3  nmin =  9

Finished!
\end{verbatim}
\normalsize
